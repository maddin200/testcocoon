\section{Installation of \TestCocoon\ with \VisualDSP\ support}

Download and install the version 5.2.6 of \TestCocoon\ from \url{http://www.testcocoon.org/download.html}.
Verify that the installation program \TestCocoon\ is detecting the presence of \VisualDSP. (cf. picture)

\InsertPictureDescription{visualdsp_install_fig}{\VisualDSP\ installation}

\section{Instrumenting an application}
\subsection{Using \CoverageScanner\ library}

4 Functions of \CoverageScanner\ library are used in the sample:
\begin{itemize}
\item \verb$__coveragescanner_save()$: this function saves the execution report (file extension \verb$.csmes$).
           This function should be called after each test execution.
\item \verb$__coveragescanner_clear()$: clears coverage information into the RAM. Should be called before running a test.
\item \verb$__coveragescanner_filename()$: Set the location of the report file.
\item \verb$__coveragescanner_set_custom_io()$: permits to specify I/O functions which are used to generate the report file.
           Only 3 functions needs to be defined: 
      \begin{itemize}
           \item \verb$fopenappend$: open the report file in append mode. \\
                                     \forExample: to use a RS232 interface instead of a file: this function can be used to setup the communication link.
           \item \verb$fputs$: write a string into the report file. \\
                                     \forExample: to use a RS232 interface instead of a file: this function can be used to write bytes to the interface.
           \item \verb$fclose$: close the communication link. \\
                                     \forExample: to use a RS232 interface instead of a file: this function can be used to release the resource of the communication link.
      \end{itemize}
\end{itemize}

\subsection{Building an application}

To compile a project with code coverage support, set the environment variable \verb$COVERAGESCANNER_ARGS$
to the list of \CoverageScanner\ command line switches. \\

Example:
\begin{itemize}
\item \verb$COVERAGESCANNER_ARGS=--cs-on$: to activate the code coverage analysis with code coverage hit.
\item \verb$COVERAGESCANNER_ARGS=--cs-on --cs-count$: to activate the code coverage analysis with code coverage count.
\item \ldots
\end{itemize}

Open \VisualDSP\ IDE and rebuild the complete project. A \verb$.csmes$ file called {\itshape appname.dxe.csmes} should be created after the build process.

\bigskip\par
The following batch script can be used as example:\\
\begin{verbatim}
set COVERAGESCANNER_ARGS="--cs-on"

call "%ProgramFiles%\Analog Devices\VisualDSP 5.0\VDSP.bat"
"%ProgramFiles%\Analog Devices\VisualDSP 5.0\System\Idde.exe"
\end{verbatim}

\section{Example}

\subsection{Sample provided}
The sample provided is a simple example working with \VisualDSP\ simulator. \\

Proceed as follows:
\begin{enumerate}
\item Click on \verb$DSP.bat$, \VisualDSP\ should start.
\item Open the \verb$test$ sample, and recompile it.
\item A file \verb$test.dxe.csmes$ should be recreated.
\item Open \CoverageBrowser\ and open the file \verb$test.dxe.csmes$.
\item The source code should be visible, and the instrumented lines should be gray (unknown state).
\item Execute in the application, a file \verb$cov.csexe$ should be created.
\item Using \CoverageBrowser\ click on \GUImenu{Load execution report}, enter a execution title and select
      the execution report \verb$cov.csexe$. Click on "Import".
\item The execution report should be imported and the code parts executed should be marks as green.

\end{enumerate}


\subsection{Sample of source code}
\PrintLongSource{visualdsp.c}
